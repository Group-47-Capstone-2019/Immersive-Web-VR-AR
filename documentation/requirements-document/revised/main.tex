\documentclass[onecolumn, draftclsnofoot,10pt, compsoc]{IEEEtran}
\usepackage{graphicx}
\usepackage{url}
\usepackage{setspace}
\usepackage{abstract}
\usepackage{longtable}
\usepackage{geometry}
\geometry{textheight=9.5in, textwidth=7in}
\parindent = 0.0 in
\parskip = 0.1 in

% 1. Fill in these details
\def \CapstoneTeamName{WebXR Team}
\def \CapstoneTeamNumber{47}
\def \GroupMemberOne{Jonathan Jones}
\def \GroupMemberTwo{Evan Brass}
\def \GroupMemberThree{Brooks Mikkelsen}
\def \GroupMemberFour{Tim Forsyth}
\def \GroupMemberFive{Brandon Mei}
\def \CapstoneProjectName{Creating Immersive Experiences on the Web using VR and AR}
\def \CapstoneSponsorCompany{Intel}
\def \CapstoneSponsorPerson{Alexis Menard}

% 2. Uncomment the appropriate line below so that the document type works
\def \DocType{		        %Problem Statement
				Requirements Document
				%Technology Review
				%Design Document
				%Progress Report
				}
			
\newcommand{\NameSigPair}[1]{\par
\makebox[2.75in][r]{#1} \hfil 	\makebox[3.25in]{\makebox[2.25in]{\hrulefill} \hfill		\makebox[.75in]{\hrulefill}}
\par\vspace{-12pt} \textit{\tiny\noindent
\makebox[2.75in]{} \hfil		\makebox[3.25in]{\makebox[2.25in][r]{Signature} \hfill	\makebox[.75in][r]{Date}}}}
% 3. If the document is not to be signed, uncomment the RENEWcommand below
%\renewcommand{\NameSigPair}[1]{#1}

%%%%%%%%%%%%%%%%%%%%%%%%%%%%%%%%%%%%%%%
\begin{document}
\begin{titlepage}
    \pagenumbering{gobble}
    \begin{singlespace}
    	%\includegraphics[height=4cm]{coe_v_spot1}
        \hfill 
        % 4. If you have a logo, use this includegraphics command to put it on the coversheet.
        %\includegraphics[height=4cm]{CompanyLogo}   
        \par\vspace{.2in}
        \centering
        \scshape{
            \huge CS Capstone \DocType \par
            {\large\today}\par
            \vspace{.5in}
            \textbf{\Huge\CapstoneProjectName}\par
            \vfill
            {\large Prepared for}\par
            \Huge \CapstoneSponsorCompany\par
            \vspace{5pt}
            {\Large\NameSigPair{\CapstoneSponsorPerson}\par}
            {\large Prepared by }\par
            Group \CapstoneTeamNumber - WebPhysicsVR\par
            % 5. comment out the line below this one if you do not wish to name your team
            %\CapstoneTeamName\par 
            \vspace{5pt}
            {\Large
                \NameSigPair{\GroupMemberOne}\par
                \NameSigPair{\GroupMemberTwo}\par
                \NameSigPair{\GroupMemberThree}\par
                \NameSigPair{\GroupMemberFour}\par
                \NameSigPair{\GroupMemberFive}\par
            }
            \vspace{20pt}
        }
        %\renewcommand{\abstracttextfont}{\sffamily}
        \begin{abstract}
        % 6. Fill in your abstract
        This document is a Software Requirements Specification (SRS) that outlines the purpose, scope, and technical requirements of the WebPhysicsVR immersive web experience hosted by the Intel 01.org open source portal. This document describes the website functionality and provides a tentative timetable for the fulfillment of project goals and requirements.
        \end{abstract}     
    \end{singlespace}
\end{titlepage}
\newpage
\pagenumbering{arabic}
\tableofcontents
% 7. uncomment this (if applicable). Consider adding a page break.
%\listoffigures
%\listoftables
\clearpage

\begin{longtable}{ |p{0.1\linewidth}|p{0.4\linewidth}|p{0.4\linewidth}| }

\hline
\textbf{Section} & \textbf{Original} & \textbf{New} \\
\hline
%----------
1.3.3
&
\begin{itemize}
    \item WebXR API
    \item WebGL 2.0
    \item (Some physics API)
    \item If needed:
    \begin{itemize}
        \item Web Workers
        \item Web Assembly (for physics / processing intensive code or porting non-JavaScript libraries)
    \end{itemize}
\end{itemize}
& 
\begin{itemize}
    \item Add mention of Cannon.js, Three.js
    \item Remove use of web workers and web assembly
\end{itemize} \\
\hline

%----------
1.4
&
\begin{itemize}
    \item XR : Family of hardware devices capable of Virtual Reality and Augmented Reality.
    \item API : Application Program Interface.
    \item HMD : Head mounted display. There is an optic lens for each eye which runs at 90Hz.
    \item WebXR : Open source web API for accessing virtual reality (VR) and augmented reality (AR) devices, including sensors and head-mounted displays.
    \item WebGL : Open graphics API.
    \item Web Worker : API for performing and communicating between parallel computation on the Web.
\end{itemize}
&
\begin{itemize}
    \item Remove reference to web workers, since we did not use them.
\end{itemize} \\
\hline

%----------
2.1
&
\begin{itemize}
    \item Users will be able to interact with objects in the environment through their controller (or simulated/virtual controller)
    \begin{itemize}
        \item Translate objects
        \item Scale objects
        \item Rotate objects
        \item Pick up/grasp objects
        \item Drop/throw objects
    \end{itemize}
\end{itemize}
& 
\begin{itemize}
    \item Removed the requirement to scale objects.
    \item Giving users the ability to scale objects worked against a good user experience as scaling could go wrong in many ways.
    \item User-scaled objects often got in the way of UI elements and in some cases removed the ability to scale objects down entirely.
    \item Without a second button on the daydream controller, scaling had to be done using in scene UI elements.
\end{itemize} \\
\hline

%----------
2.1
&
\begin{itemize}
    \item User can "pause/play". 
    \begin{itemize}
        \item During pause mode, objects are not animated and are not affected by the physics engine.
        \item During play mode, objects are animated and are affected by physics.
    \end{itemize}
    \item User can "freeze" objects, stopping all kinematic forces that the object is experiencing
\end{itemize}
& 
\begin{itemize}
    \item As with the scaling issue above, we were severely limited by the single button functionality that the controllers could provide.
    \item Without a dedicated GUI for this sort of project, we decided that it was more important to place our focus on polishing the VR components rather than struggling with out-of-scope additions.
\end{itemize} \\
\hline

2.5
&
\begin{itemize}
    \item Website will be compatible with (Enter list of devices here)
\end{itemize}
&
\begin{itemize}
    \item Added a list of compatible devices.
\end{itemize} \\
\hline

2.5
&
\begin{itemize}
    \item Website will run on multiple devices and operating systems (Windows, Android, Mac)
\end{itemize}
&
\begin{itemize}
    \item WebXR is not compatible with iOS devices.
\end{itemize} \\
\hline

\end{longtable}

% 8. now you write!
\section{Introduction}
\subsection{Purpose}
The purpose of this project is to provide a technical demonstration of the WebXR API capabilities. It will be an example of how WebXR can be used to create an entertaining, interactive VR experience within a web browser. It will have a focus on education and act as a virtual physics lab.

\subsection{Scope}
WebPhysicsVR will be an interactive physics simulation that utilizes the WebXR API. Its primary use is as tech demo and a platform for education.

As a tech demo, our project will incorporate all the major features of WebXR. We'll want to faithfully present the physical concepts in an educational and engaging fashion while also focusing on designing a quality WebXR experience.  

At a minimum, users should be able to visualize the effects of gravity.  If we have time to do so, we could visualize other forces like adding friction between objects or giving certain objects magnetism as well as mass.  

We intend to support a spectrum of VR devices though we will only test on a few selected devices that we have access to.
\subsection{Product overview}
%\subsubsection{Perspective}
\subsubsection{User Interfaces}
Users with compatible hardware will be able to experience an immersive/enhanced version of the website right when they access it, with a click of a button.

Within the VR experience, we will have interfaces for the user to manipulate the objects and physical constants. These interfaces will be displayed in the most contextually meaningful location in the environment.  This might mean different floating icons around the 3d objects, floating interfaces or control objects that are within the environment.

\subsubsection{Hardware Interfaces}
WebXR is compatible with a range of XR devices, web browsers and operating systems. This includes mobile devices. For the full intended experience, an HMD with body and controller tracking devices should be used.

\subsubsection{Software Interfaces}
\begin{itemize}
    \item WebXR API
    \item WebGL 2.0 (via Three.js)
    \item Cannon.js
\end{itemize}

\subsubsection{Functions}
The main functions of this website are:
\begin{itemize}
    \item VR Experience accessible without need to download client side software 
    \item Interactive physics environment
\end{itemize}
%\subsubsection{User characteristics}
%See IEEE 9.5.5
%\subsubsection{Limitations}
%See IEEE 9.5.6
\subsection{Definitions}
\begin{itemize}
    \item XR : Family of hardware devices capable of Virtual Reality and Augmented Reality.
    \item API : Application Program Interface.
    \item HMD : Head mounted display. There is an optic lens for each eye which runs at 90Hz.
    \item WebXR : Open source web API for accessing virtual reality (VR) and augmented reality (AR) devices, including sensors and head-mounted displays.
    \item WebGL : Open graphics API.
\end{itemize}
%\input{references.tex}
\section{Specific requirements}
\subsection{External interfaces}
\begin{itemize}
    \item Users may interface with the website using:
    \begin{itemize}
    \item Mobile devices with positional tracking
        \item Head mounted displays, whether they are opaque, transparent, or utilize video pass-through
        \item Fixed displays with head tracking capabilities
    \end{itemize}
    \item Users will be able to interact with objects in the environment through their controller (or simulated/virtual controller)
    \begin{itemize}
        \item Translate objects
        \item Rotate objects
        \item Pick up/grasp objects
        \item Drop/throw objects
    \end{itemize}
    \item Users must be able to change physical constants.  Potential mechanisms for doing that:
    \begin{itemize}
        \item Physical Constants Dashboard with levers/switches/dials to change the environment
        \item Floating user interfaces around the objects with icons/bars and other traditional UI features
    \end{itemize}
    \item User can spawn in a multitude of objects stored on the website through a spawn menu
        
\end{itemize}

\subsection{Functions}
\begin{itemize}
    \item Utilize all core features of WebXR including:
    \begin{itemize}
        \item Headset location and orientation
        \item Controller location and orientation
        \item Multi-platform compatibility
        \item Parsing input from XR devices
    \end{itemize}
    \item The website will load the correct VR environment depending on the hardware that the user has.
    \item Website will pause when a controller has been disconnected and resume when the connection is reestablished.
    \item Website will notify user if web browser/hardware is not compatible with WebXR
    \item Render environment and physics in real time
\end{itemize}
\subsection{Usability requirements}
Offer a way of manipulating objects in the simulation for all supported platforms.  On a phone this might mean having a virtual manipulator at a fixed location from the user's face. On devices like the Oculus Rift, this would probably mean the hand controllers.

We want to pair a non-technical-user friendly physics demonstration with a technical-user friendly code base.

\subsection{Performance requirements}
\begin{itemize}
    \item Latency must not exceed 11 milliseconds within the simulation.
    \item Re-factor to distribute to multiple web workers if needed to achieve performance requirements.
    \item Meet an average 60 frames per second, appropriately degrading the content to meet that budget on mobile phones and other low powered devices.
    \item HMDs should be running at 90 fps
\end{itemize}
%\subsection{Logical database requirements}
%Content - See IEEE 9.5.14
%\subsection{Design constraints}
\subsection{Software system attributes}
\begin{itemize}
    \item Landing page will be compatible with all evergreen browsers
    \item Website will be compatible with browsers that support WebXR
    \item Website will be compatible with
    \begin{itemize}
        \item Magic Window compatible mobile devices using compatible browsers.
        \item Google Daydream
        \item HTC Vive
    \end{itemize}
    \item Website will run on multiple devices and operating systems (Windows, Android, Mac)
\end{itemize}
\section{Gantt Chart}
\includegraphics[width=\linewidth]{gantt.png}

\end{document}
