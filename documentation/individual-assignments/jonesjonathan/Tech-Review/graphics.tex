\section{WebGL Frameworks - Graphics}
For most graphics programmers, OpenGL is the tool of choice. It works universally across most systems and is capable of just about anything that can be thought of. OpenGL though, is not embedded in web browsers, without the use of plugins. For that, there is WebGL; a cross-platform, royalty-free API used to create 3D graphics in a Web browser \cite{r6}. WebGL by itself can do amazing things. These things are hard to program though and have already been done by others on the web. VR, especially, is difficult to implement but these open source libraries have taken care of the boilerplate already. The graphical APIs explored in this document all use WebGL as their backend.

\subsection{ThreeJS}
Three.js is a lightweight cross-browser JavaScript library/API used to create and display animated 3D computer graphics on a Web browser. Three.js scripts may be used in conjunction with the HTML5 canvas element, SVG or WebGL \cite{r7}. The library provides Canvas 2D, SVG, CSS3D and WebGL renderers \cite{r8}. Three.js has been used to make hundreds of graphical simulations on the web and a large number of them include the WebVR API. Three.js is feature heavy and includes animation, shaders, effects, objects, debugging, virtual reality and more. It is compatible on all modern browsers and can even run on older browsers as it is able to fall back to one of its multiple renderers if one is not available on the current browser \cite{r9}.

Three.js has an incredible amount of documentation and a vibrant community that is constantly making improvements to the API. It has the most documentation out of the three graphical APIs that are discussed here. There are hundreds of examples and demos on the Three.js website that can be used for reference when developing this project. 

\subsection{BabylonJS}
BabylonJS is a graphical web framework for WebGL that was developed by Microsoft. It has a strong focus on the development of 3D games. It is about as feature heavy as Three.js and includes some features not found in the other. BabylonJS is relatively new and is much older than Three.js. It has an integrated physics engine that uses cannon.js and oimo.js as well as an integrated spatial audio system that uses WebAudio. Essentially, this framework is able to take care of most of the features that need to be implemented by itself. BabylonJS is highly optimized and looks to perform well on most systems \cite{r10}.

Like ThreeJS, Babylon has an amazing community of developers working on it who are always actively improving the framework. It is extensively documented and also boasts many demos and tutorials.

\subsection{Blend4Web}
Blend4Web is an open source, free GPLv3 licensed WebGL graphical framework that has a built in material editor with heavy support for Blender \cite{r11}. This is the most complete framework out of this list; it includes everything that Three.js and BabylonJS have and more. What is most ideal about this framework is how it was created with Blender in mind. Asset development is hard, Blender makes it easy, and Blend4Web makes implementation of Blender products even easier. 
Like Babylon, Blend4Web has an integrated physics engine, spatial audio systems and support for VR animation \cite{r11}. It has been described as “a tool for interactive 3D visualization on the Internet” \cite{r11}. There are lots of examples of Blend4Web projects and many pages of documentation on the web. Still, it remains lesser known than the previously listed frameworks which unfortunately can play into its usefulness as there is still a much better community tied to the other two.