\section{Physics API}
Physics simulations are very math intensive and require a lot of calculations behind the scenes. When VR is factored in, the process becomes much more difficult. Interactive actions like grabbing, swinging and throwing objects extends beyond simple 3D physics. There are a variety of physics engines that have been developed with WebVR in mind. Each brings its own advantages and disadvantages to the table. The physics engine used for this project needs to have VR support built in, be well documented, efficient and extensive in its features.

\subsection{Cannon.js}
Cannon.js is a lightweight 3D web physics engine written entirely in Javascript \cite{r1}. The library is open source and has been developed under an MIT license. Cannon.js primarily serves as a rigid body simulation library that is able to make objects move and interact in a realistic way \cite{r1}. It has a multitude of features that can be used to create just about any kind of physics situation. These include rigid body dynamics, discrete collision detection, friction, restitution, object constraints, and much more. There are also various shapes and collision algorithms for those shapes included in the library \cite{r1}.

Cannon.js provides just about every physics function that this project requires. It also has an enormous amount of documentation. Equally as ideal is the fact that Cannon.js is used by many of the technologies that are being considered including BabylonJS and A-Frame \cite{r2}.

\subsection{Ammo.js}
Ammo.js is a direct port of the bullet physics engine into Javascript. Bullet was originally written in C++ and is almost identical to ammo. It is open source and has been developed under a zlib license \cite{r3}. The Bullet physics engine has been a long time favorite for C++ developers. Bullet provides most, if not more features than Cannon.js does and even includes VR support. Bullet has been around for a long time and has a vibrant community, support and documentation. The same cannot be said for ammo which seems to rely on documentation for legacy Bullet. It also is not clear whether or not Ammo.js has VR supportive functions or not.

Included with Ammo.js is a compiler that is able to take Bullet code and convert it to Ammo.js friendly code. This means that it is possible to take the latest Bullet libraries and convert them into Javascript.

\subsection{Oimo.js}
Oimo.js, like Cannon.js, is a lightweight 3D physics engine for Javascript \cite{r5}. Oimo.js is a port of the OimoPhysics physics engine which was written in C++. It is open source and has been developed under an MIT license and like Ammo, it has poor documentation.  Its features include rigid body motion, fast collision and bounding volume hierarchy, contacts with friction and restitution, multiple collision geometries, joints with springs, limits and motors, breakable joints and constraint solvers \cite{r5}. Oimo.js is a powerful yet simple 3D graphics library. The WebGL graphical framework, BabylonJS, uses Oimo.js in conjunction with Cannon.js under the hood for its integrated physics. This makes up for the lack of documentation on the Oimo.js website as it is already provided in the Babylon.js documentation.