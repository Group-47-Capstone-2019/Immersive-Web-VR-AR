\documentclass[10pt,peerreview]{IEEEtran}
\usepackage[utf8]{inputenc}

\title{Problem Statement}
\author{Brooks Mikkelsen}
\date{October 2018}

\begin{document}

\maketitle

\section{Abstract}
Web virtual reality and augmented reality is a technology that is just getting started and is very cutting edge. In the past few years, a standard has come out specifying an API for creating web apps that use virtual reality and augmented reality. Because this is such a new field, there are not many projects that provide examples to other developers trying to use this technology. The goal of our project is to create an app that provides a showcase of the things web VR and AR can do and a reference for these developers. As of now, there is no specification on the actual VR/AR app we will be creating; that will be defined as we research the technology. We will also need to document research we have done and problems we encounter so that others can use it as a reference for building web AR/VR apps. 

\section{Problem Definition}
Virtual reality (VR) and augmented reality (AR) are both still in the early stages of production. Because the industry is so new, entrepreneurs and industry leaders are still coming up with new ways virtual reality and augmented reality can be harnessed to create great products. Currently, most products require being downloaded onto a computer before being ran. Examples of this method of distribution include steam games that support virtual reality and product training simulations that are downloaded to a computer. While this is great for larger apps that need gigabytes of textures and 3D object files downloaded before they can even run, many smaller AR or VR apps might not need as much overhead. 

This is where Web virtual reality and augmented reality comes into play. New web specifications have been created that call for a virtual reality web API (application programmer interface – the tools developers use to build applications on top of an existing framework) called the WebVR API. The idea is that a user could navigate to a website via a typical web browser such as Google Chrome, Safari, or Firefox and use their computer’s graphics card to render VR or AR to a headset such as the Oculus Rift or Microsoft HoloLens. This would allow users to quickly check out new AR and VR experiences in the same way that streaming makes watching videos online much faster and easier. 

Currently, because the WebVR API is in such early stages (it isn’t even available for some browsers yet), there are not a lot of examples for developers to use as a reference point while creating new experiences. Without a strong set of open source examples and articles where developers can learn from others’ mistakes before them, creating WebVR projects becomes a lot more time consuming. There is also a lot more troubleshooting and research involved in the process before developers can create a new experience. 

\section{Proposed Solution}
This is where our team comes in. Our goal is to make a sample project that will demonstrate how to use the API to other developers. We want to include as many of the common features as possible so that people trying to make projects in the future can reference our project for a wide variety of use cases. This website will eventually be posted on Intel’s open source portal, 01.org so that people in the open source community can try out the technology and see the source code if they are interested in creating a web virtual reality or augmented reality project of their own. 

There are several research components to this project that must be done before we will know what we want to do in specific detail. This is because if we want to make an app that will be used as a reference point by other developers, it needs to set a good example for them. As our client noted, we will first need to research the best way to experience the AR and VR content. We will then need to research how to implement this method. We will also need to research what value the benefits brought by adding AR or VR functionality are, so we can create a guide helping developers decide whether it’s worth their time and money to implement the functionality in their app. 

As we do this research, we will work with our client to develop the actual app we are creating as an example to the community. In this respect, we will get some degree of freedom in choosing the deliverable project. 

\section{Performance Metrics}
The app that we end up creating has a few requirements so far, although this list will grow as our research develops. The first is that it uses the new WebVR API and utilizes enough of the API’s functionalities that someone else could use the app as a reference. The second is that that it is a usable app with a friendly user interface and user experience, including the surrounding website. The second is that it is open source and well enough document that others can follow it to create their own web augmented reality or virtual reality applications based on this one. Although there are requirements for the website that will be our end product, there is not any specification about whether the virtual reality experience must be a video game or simulation, for example.

Although the end product is an important part of the deliverable, the research we did to come up with the project and all of the issues we encountered along the way are also very valuable because of the nature of the project. Therefore, we will need to be very thorough when documenting this research and these issues so that other people can come behind us and use what we have learned to create other great web AR and VR products. 

As of yet, there are no quantifiable performance metrics that need to be met for the project, but these will be updated as our research progresses and we begin to have a better understanding of what our final deliverable will be. 

\end{document}
