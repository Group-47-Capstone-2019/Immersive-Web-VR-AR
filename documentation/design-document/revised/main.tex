  \documentclass[onecolumn, draftclsnofoot,10pt, compsoc]{IEEEtran}
\usepackage{graphicx}
\usepackage{url}
\usepackage{setspace}
\usepackage{abstract}
\usepackage{geometry}
\usepackage{longtable}
\geometry{textheight=9.5in, textwidth=7in}
\parindent = 0.0 in
\parskip = 0.1 in

% 1. Fill in these details
\def \CapstoneTeamName{WebXR Team}
\def \CapstoneTeamNumber{47}
\def \GroupMemberOne{Brooks Mikkelsen}
\def \GroupMemberTwo{Evan Brass}
\def \GroupMemberThree{Jonathan Jones}
\def \GroupMemberFour{Brandon Mei}
\def \GroupMemberFive{Tim Forsyth}
\def \CapstoneProjectName{Creating Immersive Experiences on the Web using VR and AR.}
\def \CapstoneSponsorCompany{Intel}
\def \CapstoneSponsorPerson{Alexis Menard}

% 2. Uncomment the appropriate line below so that the document type works
\def \DocType{		Design Document
				%Requirements Document
				%Technology Review
				%Design Document
				%Progress Report
				}
			
\newcommand{\NameSigPair}[1]{\par
\makebox[2.75in][r]{#1} \hfil 	\makebox[3.25in]{\makebox[2.25in]{\hrulefill} \hfill		\makebox[.75in]{\hrulefill}}
\par\vspace{-12pt} \textit{\tiny\noindent
\makebox[2.75in]{} \hfil		\makebox[3.25in]{\makebox[2.25in][r]{Signature} \hfill	\makebox[.75in][r]{Date}}}}
% 3. If the document is not to be signed, uncomment the RENEWcommand below
%\renewcommand{\NameSigPair}[1]{#1}

%%%%%%%%%%%%%%%%%%%%%%%%%%%%%%%%%%%%%%%
\begin{document}
\begin{titlepage}
    \pagenumbering{gobble}
    \begin{singlespace}
    	%\includegraphics[height=4cm]{coe_v_spot1}
        \hfill 
        % 4. If you have a logo, use this includegraphics command to put it on the coversheet.
        %\includegraphics[height=4cm]{CompanyLogo}   
        \par\vspace{.2in}
        \centering
        \scshape{
            \huge CS Capstone \DocType \par
            {\large\today}\par
            \vspace{.5in}
            \textbf{\Huge\CapstoneProjectName}\par
            \vfill
            {\large Prepared for}\par
            \Huge \CapstoneSponsorCompany\par
            \vspace{5pt}
            {\Large\NameSigPair{\CapstoneSponsorPerson}\par}
            {\large Prepared by }\par
            Group \CapstoneTeamNumber\par
            % 5. comment out the line below this one if you do not wish to name your team
            %\CapstoneTeamName\par 
            \vspace{5pt}
            {\Large
                \NameSigPair{\GroupMemberOne}\par
                \NameSigPair{\GroupMemberTwo}\par
                \NameSigPair{\GroupMemberThree}\par
                \NameSigPair{\GroupMemberFour}\par
                \NameSigPair{\GroupMemberFive}\par
            }
            \vspace{20pt}
        }
        %\renewcommand{\abstracttextfont}{\sffamily}
        \begin{abstract}
        This document is a software design specification that is intended to outline the various components used in the creation of the WebPhysicsVR simulation. There is a list of requirements that must be met for this project, their technical solutions, descriptions of those technologies and rational for using them.
        \end{abstract}     
    \end{singlespace}
\end{titlepage}
\newpage
\pagenumbering{arabic}
\tableofcontents
% 7. uncomment this (if applicable). Consider adding a page break.
%\listoffigures
%\listoftables
\clearpage

% ------------- update table -------------------
\begin{longtable}{ |p{0.1\linewidth}|p{0.4\linewidth}|p{0.4\linewidth}| } 
 \hline
 \textbf{Section} & \textbf{Original} & \textbf{New} \\
 \hline
 2.1.2 & Objects in the scene must be interactive through translation, scaling, rotation and collision. & Objects in the scene must be interactive through translation, rotation and collision.\\
 \hline
 
 2.4.1 &
 
 Stretch: Content may be "installable" to make it available offline. & 
 (Removed)
 \begin{itemize}
     \item Content is technically installable via use of the dev environment and localhost but an installable feature from the website itself is a stretch goal that is not needed for this project.
 \end{itemize}\\
 \hline
 
 2.5.1 &
 Users must be able to choose to display information about an object, such as velocity, acceleration, and mass. & Users must be able to visualize velocity vectors on moving objects in the kinematics scene.
 \begin{itemize}
     \item The original idea was to use a GUI interface such as DATGUI VR but an API such as this is constructed for the WebVR API and not WebXR.
     \item We settled on using a custom text display system but it was not optimized enough to handle updates for every frame. For that reason, attribute displays were mostly omitted for objects.
 \end{itemize}\\
 \hline
 
 2.5.2 & 
 A menu must be available for users to switch stations or select objects. & Users can interact with the scene to traverse through it and into other scenes.
 \begin{itemize}
     \item Rather than traversing through the scenes via a user interface, we found a better alternative in users actually interacting with objects in the scene to traverse.
 \end{itemize}\\
 \hline
 
 2.6.1 &
 A scene can be paused/frozen. &
 (Removed)
 \begin{itemize}
     \item It no longer seemed practical to add this functionality since we were limited to a single button press on a controller. We decided that the user experience would be too low to trying to aim their controller at an interface on the wall to pause the scene.
 \end{itemize}\\
 \hline
 
 2.6.2 &
 An object can be paused/frozen. & 
 (Removed)
 \begin{itemize}
     \item It no longer seemed practical to add this functionality since we were limited to a single button press on a controller. We decided that the user experience would be too low to trying to aim their controller at an interface on the wall to pause the scene.
 \end{itemize}\\
 \hline
 
 2.7.4 &
 Users must be able to interact with objects in the scene using a mobile touchscreen interface. & 
 Users will be able to select objects in the scene using a mobile touchscreen interface.
 \begin{itemize}
     \item Unfortunately, the WebXR API does not support drag and drop events for touchscreens, so we are only able to implement select functionality (tapping objects).
 \end{itemize}\\
 \hline
 
 2.8 Web Hosting &
 Users must be able to host directly from the GitHub repository. &
 The project will be published on a website using a Now deployment.
 \begin{itemize}
     \item We are not using GitHub pages, instead we're using Now deployment. 
 \end{itemize}\\
 \hline
 
 2.9 Version Control &
 Users must be able to edit, push, pull request, and the changes are live.  & 
 (Removed)
 \begin{itemize}
     \item This was meant for the developers, not the users. Does not need to be in the design document.
 \end{itemize}\\
 \hline
 
 3.3 
 &
 \begin{itemize}
    \item In this section, ways of optimizing our site were explored.
    \item At the time, we were unsure of the web capability for VR experiences and sought ways to prevent exceeding our frame budget.
 \end{itemize}
 &
 (Removed)
\begin{itemize}
    \item During the course of the project, we quickly discovered that tools like ThreeJS and the WebXR API took care of multithreading and optimization for us.
    \item There was no need on our end to implement performance boosting tools.
\end{itemize} \\
\hline

3.4 
&
Service Workers expose the browser's caching system to developers. Our service worker can prime the browser cache with VR assets and respond to asset requests when disconnected.  To describe to the browser what modules and assets will be needed later, we will use HTML meta elements.  On parallel serving protocols like (G)QUIC and HTTP 2 the browser can load those assets over a low priority stream or maintain a connection to the server while waiting for the app to request those assets.  We will also use responsive images to give the browser multiple options for an image source and it will pick whichever one would load best in terms of resolution, supported format, etc.
&
Many 3D and 2D assets will be used to style this site from a variety of third party sources. In the interest of providing a smooth user experience, there needs to be some overhead when loading assets into a scene. For this task, we'll employ a script of our own to asynchronously load into our scenes the assets that we need. During the time in which the scene assets are loading, a loading screen will be displayed to keep the user experience clean.
\begin{itemize}
    \item Service workers and web assembly were an overcomplication of a simple task. Loading assets into the scene was easily achieved through the use of the ThreeJS API. Caching our objects was as easy as keeping a reference to their loaded instances between scenes.
\end{itemize} \\
\hline

3.5
&
Spatial audio is required to achieve full immersion into the physics simulation. Audio clips must be played when events happen in the scene.
\begin{itemize}
    \item This section originally handled the Audio component, which has been omitted from the final product.
\end{itemize} 
&
(Removed)
\begin{itemize}
    \item This project no longer uses audio components.
\end{itemize} \\
\hline

3.6
&
The web hosting services must able to host static web pages for GitHub users to publish content. GitHub Pages can also be served securely which is a requirement for using service workers.
&
The site will be hosted on the NOW hosting service provided by ZEIT.CO for testing and deployment.
\begin{itemize}
    \item We opted for using NOW deployment instead of github pages as it offers more testing capability and a wider selection of deployments to view throughout development.
\end{itemize} \\
\hline

3.8
&
\begin{itemize}
    \item Originally referenced Jekyll as the publishing system.
\end{itemize}
&
\begin{itemize}
    \item Changed Jekyll to ZEIT as that has been our publishing system throughout deployment.
\end{itemize} \\
\hline

3.9
&
\begin{itemize}
    \item Old version specified typescript as the primary language for this project.
\end{itemize}
&
\begin{itemize}
    \item Types are not available for WebXR
    \item Changed to Javascript
\end{itemize} \\
\hline

3.12
&
\begin{itemize}
    \item Listed AFrame as a component.
\end{itemize}
&
\begin{itemize}
    \item AFrame was never used. Unnecessary component.
    \item Removed AFrame.
\end{itemize} \\
\hline

3.13
&
\begin{itemize}
    \item Listed AFrame as a component.
\end{itemize}
&
\begin{itemize}
    \item AFrame was never used. Unnecessary component.
    \item Removed AFrame.
\end{itemize} \\
\hline

3.14.3 Personal Computer &
The  product  should  be  playable  on  a  computer  browser  with  the  ability  to  switch  to  a  full  VR experience if the appropriate hardware is detected.
&
The product can be run on a computer browser with the ability to switch to a full VR experience if the appropriate hardware is detected.
\begin{itemize}
    \item Removed mouse interactions since the WebXR APIs input source did not support mouse clicks. Since this project is focusing on the WebXR API, we decided it was not necessary.
\end{itemize}\\
\hline
 %%%%% start of brooks changes section 4
4.3
&
A physics simulation should have sound for it to be immersive in the VR world. The applicaiton will use the WebAudio API to deliver event driven spatial sounds to the simulation. HowlerJS is an API for WebAudio that makes its implementation easier. Spatial audio is supported as well as event playback.
& 
(Removed)
\begin{itemize}
    \item This requirement was removed so we could better focus on demonstrating the capabilities of WebXR.
    \item Therefore, this Design View was removed as well.
\end{itemize}\\
\hline


4.4
&
The hosting will need to host static web pages for GitHub users, blogs, and project documentation. We will be using GitHub pages as there are no required databases to setup and no server to configure. It also needs to serve all of the project site from a personal URLs that is tied to an organization or GitHub account. The GitHub pages will automatically build and deploy our site. 
& 
Zeit Now will be used to host the website as we develop it so that our progress can be shared with the client and other team members. It allows us to set up continuous deployment for static websites easily. Zeit now also allows us to create a unique and lasting deployment for each commit that is pushed to GitHub. This will allow us to share individual progress with stakeholders in the project.
\begin{itemize}
    \item Changed from using GitHub to host the website to Zeit Now. 
    \item Now offered us better customization and the ability to create individual deployments per commit, so we could reference a specific version of the website. 
\end{itemize} \\
\hline

4.6 
&
The hosting sites must converts the files into a static website that can be deliver with any standard web servers. That means it needs to automatically generate the HTML code in the background if there are changes that are made to the files. Jekyll is a engine of GitHub Pages that can be use with each other, while it provides updates and HTML rendering.
&
(Removed)
\begin{itemize}
    \item Bundler - Parcel - Does this, and we tested hosting our project from another server called Caddy
    \item This optional requirement has been removed. 
    \item It was originally for if we wanted to blog (Jekyll's primary goal) about our experiences using WebXR.
\end{itemize} \\
\hline

4.7 
&
The application must run in modern browsers that have implemented the WebXR API. That means that the code must compile down to JavaScript (or be written in JavaScript). We will be using TypeScript to write code for the client side application. TypeScript is a language written by Microsoft that provides static typing on top of traditional JavaScript syntax. It can be compiled down to JavaScript for use in both Node.js and in browsers.

Rust is a systems programming language founded by Mozilla for designing programs with safe memory sharing and simpler parallelism without the usual trade offs of a garbage collector or heavy runtime.
& 
We will be using JavaScript for the client side application because it is what is supported by WebXR and ThreeJS. Unfortunately, ThreeJS does not have updated TypeScript types and WebXR has no TypeScript type definitions at all, so its not practical to use for this project.
\begin{itemize}
    \item Our program is tested using Chrome which is the only browser with an implementation of WebXR
    \item We opted to use JavaScript instead of TypeScript because there are no TypeScript definitions for WebXR. 
    \item We decided against using Rust because Three.JS has excellent support for vector math, the most computationally rigorous code we wrote. 
\end{itemize} \\
\hline

4.10 
&
User interfacing needs some sort of graphical representation that users will be able to view and interact with. This will be accomplished by using Three.js along with A-Frame to create the necessary models to be used as a graphical interface. A-Frame provides primitives that simplifies creating objects, such as a flat plane, that can be used in the creation of user menus or graphical displays. Three.js can also achieve this at a lower level, which will be used in adding finer details.
& 
\begin{itemize}
    \item Since A-Frame simplifies the task of using WebXR API, we decided to make our user interface manually. 
    \item This is so we can stick to the primary goal of demonstrating how to use WebXR. 
\end{itemize} \\
\hline
 
4.11 
&
Many of the graphical user interfaces will need some sort of functionality, otherwise they will just be objects that do nothing. The main user interface functionality is menu functionality. The user must be able to interact with a menu to select an experiment station, pause and play objects or the environment, spawn or delete objects, and others. Nearly everything the user can interact with requires some sort of interface. While the direct functionality may not be defined with A-Frame, A-Frame components will be used to apply the functionality to objects, in this case the interface models. 
& 
\begin{itemize}
    \item Removed references to A-Frame (see above).
    \item Added portion about technology used to integrate User Interaction into our project.
\end{itemize} \\
\hline
 
4.12
&
HMDs must be supported, therefore some sort of interfacing is needed to set up the peripherals such as the controllers and head tracking. This will be done using Gamepad API and A-Frame. A-Frame provides a very high level built in component, laser-controls, that provides tracked controls with a laser or ray cursor. Gamepad API is a much lower level API that these A-Frame components are built from. Gamepad API will be used for finer details in creating specific controls and gamepad setup.

Mobile devices should be supported, in particular those with Google Daydream support. The high level A-Frame component, laser-controls, includes support for Daydream controls. This means it will be possible to set up support for various control schemes at once by using the laser-controls component. Touchscreen should also be supported for devices without Daydream support. This will be achieved by using various A-Frame components with touchscreen support, such as the look-controls component that allows the camera to be moved by using a touchscreen with touch-drag support.

Computers should be supported if they are running in a compatible web browser. It should display the environment in their web browser with the ability to move with a keyboard and interact with the environment by using a mouse. This will be achieved by using several A-Frame components: wasd-controls, look-controls, and cursor. The component wasd-controls will provide the ability to move an object by using the wasd keys, most likely the camera, look-controls allows the user to rotate an object when the mouse is moved and cursor will hover and click functionality for the user to interact with other entities.
& 
\begin{itemize}
    \item Removed references to A-Frame (see above)
    \item We used the WebXR controller API instead.
    \item Removed requirements about compatibility with non-XR devices such as laptops. This is because we wanted to focus the scope of this project on WebXR.
\end{itemize} \\
\hline
 
\hline

 5.2.1
 & 
 Web Workers will be efficient utilization of modern multi-core CPU's for better battery life and response times
    
 &
 (Removed)
 \begin{itemize}
     \item Performance management is no longer necessary since the APIs we are taking advantage of, ThreeJS and WebXR, handle optimization automatically.
 \end{itemize} \\
 \hline
 
 5.2.2 
 & 
 WebAssembly will be faster and more efficient execution with a smaller runtime and response times. 
    
 &
 (Removed)
 \begin{itemize}
     \item Performance management is no longer necessary since the APIs we are taking advantage of, ThreeJS and WebXR, handle optimization automatically.
 \end{itemize} \\
 \hline
 
  5.2.3
 & 
 Service Worker will browser cache priming, pruning, and offline service.
    
 &
 (Removed)
 \begin{itemize}
    \item Our site is static which allows it to be cached well by HTTP's default caching algorithms.
    \item Performance management is no longer necessary since the APIs we are taking advantage of, ThreeJS and WebXR, handle optimization automatically.
 \end{itemize} \\
 \hline
 
 5.4 Audio
 &  
 HowlerJS Handles audio events and playback on call.

 & 
(Removed)
 \begin{itemize}
     \item HowlerJS is not part of our main WebXR features.
 \end{itemize} \\
 \hline
 
 5.5 Web Hosting
 & 
GitHub Pages will be used for hosting static websites and able to edit, push, and make live changes directly from a GitHub repository.
 & 
 \begin{itemize}
    \item Using Zeit Now to host website instead of GitHub.
    \item This is because it provides better support for hosting multiple versions of the site at the same time.
 \end{itemize} \\
 \hline
 
 5.6 Blog Publishing System
 & Jekyll will be use for taking content and spits out static website to be ready to serve a web hosting server.Jekyll is part of GitHub Pages, which can be used to host sites from the GitHub repositories.

 & 
 (Removed)
 \begin{itemize}
    \item No longer require for our WebXR experiences.
 \end{itemize} \\
 \hline
 
 5.8.1 TypeScript
 &
TypeScript will be used to write all client side code for the app, except for that interacting with the WebXR API.
 & 
JavaScript will be used to write all client side code for the app.
 \begin{itemize}
    \item TypeScript was no longer viable to user due to WebXR not having type definitions and ThreeJS types being out of date.
 \end{itemize} \\
 \hline
 
 5.8.2 Rust
 &
Safe systems programming with a small runtime and a WebAssembly cross-compilation target.
 & 
 (Removed)
 \begin{itemize}
    \item We decided against using Rust because Three.JS has excellent support for vector math, the most computationally rigorous code we wrote. 
 \end{itemize} \\
 \hline
 
 5.11.2 A-Frame
 &
A-Frame will assist Three.js in creating these menus by using A-Frame specific tools, such as primitives.
 & 
 (Removed)
 \begin{itemize}
    \item A-Frame was not utilized since we wanted the control that ThreeJS had to offer.
 \end{itemize} \\
 \hline
 
 5.12.2 A-Frame
 &
A-Frame  will  be  coupled  with  JavaScript  to  apply  functional  code  to  specific  objects,  in  this  case  the  user interface models, by using A-Frame components.
 & 
 (Removed)
 \begin{itemize}
    \item A-Frame was not utilized since we wanted the control that ThreeJS had to offer.
 \end{itemize} \\
 \hline
 
 5.13.2 A-Frame
 &
A-Frame provides built in components that aid in setting up controls for a wide range of devices. This includes HMDs, mouse and keyboard, and mobile devices with gaze interaction.
 & 
 (Removed)
 \begin{itemize}
    \item The WebXR API dealt with controls through the Gamepad API.
 \end{itemize} \\
 \hline
 
 
 6.3 Audio
 &
    There were not a lot of options for Audio APIs, still, HowlerJS stood out among its peers. Its abstraction of the WebAudioAPI and support for event driven spatial audio was much desired for this simulation. This API makes the use of theWebAudio API much easier and in the event that it is unsupported on certain browsers, it falls back to HTML
 & (Removed)
 \begin{itemize}
    \item We are no longer using audio in this project as we wanted to focus more on implementing the WebXR features.
 \end{itemize} \\
 \hline
 
 6.4 Web Hosting
 &
The web hosting services we choose to use GitHub Pages because it will offer us static web pages for GitHub users. Static web page is a web pages that can deliver the users exactly what is stored. GitHub Pages has intergrated with Jekyll to automatically update GitHub Pages servers and regenerate the site. To host a website from GitHub Pages we will need to create a repository with a repository name plus github.io. Then enter the project folder and add/push an index.html file in order to host the website easily.  
 & 
 \begin{itemize}
    \item Change GitHub to Zeit Now for static web hosting
    \item This is because it supports multiple versions of the website being deployed at the same time.
 \end{itemize} \\
 \hline
 
  6.6 Blog Publishing System
 &
Since Jekyll is intergrated with GitHub Pages and repository of GitHub. We chose to use Jekyll as a very simple, blog-aware, and static site generator for our project. While not have to worry with needless complexity and configuration, it allows us to concentrate on the content instead. 
 & (Removed)
 \begin{itemize}
    \item We are no longer using any Blog publishing system as GitHub handles our project fine and focus more on implemeting the WebXR features. 
 \end{itemize} \\
 \hline
 
   6.10 User Interface Visualization
 &
 \begin{itemize}
    \item Had references to A-Frame.
 \end{itemize}
 & (Removed)
 \begin{itemize}
    \item We decided against using A-Frame as ThreeJS gave us more control.
 \end{itemize} \\
 \hline
 
   6.11 User Interface Functionality
 &
 \begin{itemize}
    \item Removed any references to A-Frame
 \end{itemize}
 & (Removed)
 \begin{itemize}
    \item Had references to A-Frame.
 \end{itemize} \\
 \hline
 
 6.12 Design Rational - Physics
 &
 \begin{itemize}
     \item We had chosen to use Cannon in part because it was integrated with A-Frame
 \end{itemize}
 &
 \begin{itemize}
     \item Removed reference to A-Frame
 \end{itemize} \\
 \hline
 
 6.13 Design Rational - Graphics
 &
 (Three.js) has VR support and is built into the A-Frame library.
 &
 \begin{itemize}
     \item Removed reference to A-Frame
 \end{itemize} \\
 \hline
 
 6.14 Design Rational - Asset Bundler
 &
 
 &
 \begin{itemize}
     \item Added sentence about how Parcel's bundling allows us to use multiple module formats.
 \end{itemize} \\
 \hline
 
 6.15 Design Rational - Hardware Interfacing
 &
 Had a sentence about using A-Frame
 &
 \begin{itemize}
     \item Replaced that sentence with one about the hardware API functionality which was finalized in the spec.
 \end{itemize} \\
 \hline
 

\end{longtable}

% 8. now you write!
\section{Introduction}
This software design document is intended for the WebXR Senior Capstone Project for Oregon State University. The stakeholder of this project is Alexis Menard of Intel's open source software division. The goal of the project is to provide an example project for other developers interested in using the up and coming WebXR API built for creating VR and AR experiences on the web. Another goal of the project is to showcase an application that can be impossible to perform without a VR and AR device while progressively enhancing the experience to users. The website will serve an additional purpose as an immersive educational physics application.

We will be building a virtual reality physics simulation for use within classrooms to demonstrate how to use the API while creating something valuable for physics students at the same time. The project can be divided into individual components for the purpose of project planning, including: project setup, site hosting, physics, graphics, audio, and user interface. This document will list project requirements, then address these using design viewpoints, views, and elements, and rationales. 

\section{Design Stakeholders and their Concerns}
%\subsection{DELETEME (Sort these into their respective component section) - Requirements list}
%\subsubsection{}
%Website utilizes the core features of WebXR.
%\subsubsection{}
%Website loads the correct VR environment depending on the connected hardware.

\subsection{Physics}
\subsubsection{}
The physical constants for a scene or object can be changed.
\subsubsection{}
Objects in the scene must be interactive through translation, rotation and collision.

\subsection{Graphics}
\subsubsection{}
3D content is rendered onto the connected devices.
\subsubsection{}
Content is tailored to mobile devices when accessed through one.

\subsection{Performance}
\subsubsection{}
Simulation is rendered in real time.
\subsubsection{}
Rendering the content should fit within frame budget in VR, typically within 11 milliseconds.
\subsubsection{}
Frame rate is 60-90fps on connected HMDs.

\subsection{Client Reach}
\subsubsection{Mobile Networks and Offline Use}
Content must be accessible from mobile speed networks.
\subsubsection{Inexpensive and Available Hardware}
Content must render comfortably on cell phones manufactured later than two or three years ago.

Content must be interact-able without an external controller.

\subsection{User Interface Visualization}
\subsubsection{}
Users must be able to visualize velocity vectors on moving objects in the kinematics scene.
\subsubsection{}
Users can interact with the scene to traverse through it and into other scenes.
\subsection{User Interface Functionality}
\subsubsection{}
Users can bring objects into the scene through interfaces in the simulation.

\subsection{Hardware Interfacing}
\subsubsection{}
Users will be able to interface using mobile devices.
\subsubsection{}
Users will be able to interface using HMDs whether they are opaque, transparent or utilize video pass-through.
\subsubsection{}
Users must be able to interact with objects in the scene using position tracked controllers.
\subsubsection{}
Users will be able to select objects in the scene using a mobile touchscreen interface.
\subsubsection{}
Simulation pauses when a device has been disconnected and waits until reconnection to resume.
\subsubsection{}
Website will notify user if the web browser or hardware is not compatible with WebXR.

\subsection{Web Hosting}
\subsubsection{}
The project will be published on a website using a Now deployment.

\section{Design Viewpoints}

\subsection{Physics}
\textbf{Concerns:}  2.1.1, 2.1.2, 2.2.2, 2.3.2, 2.6.1, 2.6.2\\
\textbf{Elements:} Cannon.js\\
\textbf{Analytical Methods:} The simulation must be animated in a realistic way. Objects must interact with each other and be interactive with the user.\\
\textbf{Viewpoint Source:} Jonathan Jones

\subsection{Graphics}
\textbf{Concerns:} 2.2.1, 2.2.2, 2.4.2\\
\textbf{Elements:} Three.js\\
\textbf{Analytical Methods:} 3D content needs to be rendered onto the website and tunneled to the respective devices accessing the website. VR must supported.\\
\textbf{Viewpoint Source:} Jonathan Jones

\subsection{Installability and Caching}
\textbf{Concerns:} 2.2.2, 2.4.1, 2.4.2\\
\textbf{Elements:} Loading Script\\
\textbf{Analytical Methods:} Many 3D and 2D assets will be used to style this site from a variety of third party sources. In the interest of providing a smooth user experience, there needs to be some overhead when loading assets into a scene. For this task, we'll employ a script of our own to asynchronously load into our scenes the assets that we need. During the time in which the scene assets are loading, a loading screen will be displayed to keep the user experience clean.\\
\textbf{Viewpoint Source:} Evan Brass

\subsection{Web Hosting}
\textbf{Concerns:} 2.8.1\\
\textbf{Elements:} Now-ZEIT\\
\textbf{Analytical Methods:} The site will be hosted on the NOW hosting service provided by ZEIT.CO for testing a deployment.
\textbf{Viewpoint Source:} Brandon Mei

\subsection{Version Control System}
\textbf{Concerns:} 2.9.1\\
\textbf{Elements:} GitHub\\
\textbf{Analytical Methods:} The service must host models assets, including audio, texture, models, and video files.\\
\textbf{Viewpoint Source:} Brandon Mei

\subsection{cloud platform System}
\textbf{Concerns:} 2.8.1\\
\textbf{Elements:} Now-ZEIT\\
\textbf{Analytical Methods:} The site will be hosted on the NOW hosting service provided by ZEIT.CO for testing a deployment.
\textbf{Viewpoint Source:} Brandon Mei

\subsection{Language}
\textbf{Concerns:} Project Setup\\
\textbf{Elements:} JavaScript\\
\textbf{Analytical Methods:} The product must compile to browser-executable, bug free JavaScript code.\\ 
\textbf{Viewpoint Source:} Brooks Mikkelsen

\subsection{Asset Bundler}
\textbf{Concerns:} Project Setup\\
\textbf{Elements:} Parcel\\ 
\textbf{Analytical Methods:} The product must download all required assets on page load or before they are required to be displayed.\\
\textbf{Viewpoint Source:} Brooks Mikkelsen

\subsection{Continuous Integration}
\textbf{Concerns:} Project Setup\\
\textbf{Elements:} Circle CI\\
\textbf{Analytical Methods:} The product be written with correct and maintainable code.\\
\textbf{Viewpoint Source:} Brooks Mikkelsen

\subsection{User Interface Visualization}
\textbf{Concerns:} 2.5.1, 2.5.2\\
\textbf{Elements:} Three.js\\
\textbf{Analytical Methods:} Various user interfaces, such as a menu, are required for users to interact with the simulation. These user interfaces need some sort of graphical visualization.\\
\textbf{Viewpoint Source:} Tim Forsyth

\subsection{User Interface Functionality}
\textbf{Concerns:} 2.6.1, 2.6.2, 2.6.3\\
\textbf{Elements:} JavaScript\\
\textbf{Analytical Methods:} User interfaces need some sort of functionality for users to interact with.\\
\textbf{Viewpoint Source:} Tim Forsyth

\subsection{Hardware Interfacing}
\subsubsection{Head Mounted Display}
\textbf{Concerns:} 2.7.2, 2.7.3\\
\textbf{Elements:} Gamepad API\\
\textbf{Analytical Methods:} The product should be compatible with VR HMDs and controllers for users to interact with the environment with 6 DoF.\\
\textbf{Viewpoint Source:} Tim Forsyth

\subsubsection{Mobile Device}
\textbf{Concerns:} 2.7.1, 2.7.4\\
\textbf{Elements:} Gamepad API, Reticulum\\
\textbf{Analytical Methods:} The product should be compatible with most mobile devices, including support for those with Daydream compatibility to offer a wide variety of experiences.\\
\textbf{Viewpoint Source:} Tim Forsyth

\subsubsection{Personal Computer}
\textbf{Concerns:} 2.7.5, 2.7.6\\
\textbf{Elements:} WebXR\\
\textbf{Analytical Methods:} The product can be run on a computer browser with the ability to switch to a full VR experience if the appropriate hardware is detected.\\
\textbf{Viewpoint Source:} Tim Forsyth

\section{Design Views}

\subsection{Physics}
A physics simulation needs a physics engine to function properly. Cannon.js is a lightweight 3D web physics engine written entirely in Javascript. It provides rigid body dynamics, discrete collision detection, friction, restitution, object constraints, etc.

\subsection{Graphics}
WebGL will be used to render graphics to the connected devices and the website. The website will use a JavaScript library to implement WebGL features. Three.js is a lightweight cross-browser JavaScript library/API used to create and display animated 3D computer graphics on a Web browser. Three.js scripts may be used in conjunction with the HTML5 canvas element, SVG or WebGL. The library provides Canvas 2D, SVG, CSS3D and WebGL renderers.

\subsection{Web Hosting}
Zeit Now will be used to host the website as we develop it so that our progress can be shared with the client and other team members. It allows us to set up continuous deployment for static websites easily. Zeit now also allows us to create a unique and lasting deployment for each commit that is pushed to GitHub. This will allow us to share individual progress with stakeholders in the project. 

\subsection{Version Control Systems}
The hosting service must have a repository to sort all the files that can be accessed with a unique URL. We will be using GitHub to manage repositories and it also will be used to host open-source projects. GitHub provides access control and features such as bug tracking, feature request, and task management. 

\subsection{Cloud Platform System}
ZEIT Now is a cloud platform for serverless deployment. It enables us to host websites and web services that deploy instantly, scale automatically, and require no supervision, all with minimal configuration.

\subsection{Language}
The application must run in modern browsers that have implemented the WebXR API. That means that the code must compile down to JavaScript (or be written in JavaScript). We will be using a modern ES2018 version of JavaScript to write code for the client side application. It can be compiled down to an older version of JavaScript for use in both Node.js and in browsers.

\subsection{Asset Bundler}
The product must download all required assets on page load or before they are required to be displayed. We will be using Parcel to bundle all of our assets, including code, images, 3D object files, and textures. Parcel takes a single HTML file as an entrypoint, and then resolves all dependencies in a tree format without needing any configuration. It also supports code splitting using dynamic imports with no configuration necessary. If we need more configuration in our asset bundler, we can use Webpack, as provides all this functionality, but requires more configuration. 

\subsection{Continuous Integration}
The product must be written in a clean, easily maintainable fashion. To ensure this, we will use Circle CI to build and test our application. Circle CI is a continuous integration service that can hook into GitHub and can be configured to build and test our application after each push to GitHub and before each pull request is merged to ensure the quality of our code. 

\subsection{User Interface Visualization}
User interfacing needs some sort of graphical representation that users will be able to view and interact with. This will be accomplished by using Three.js to create the necessary models to be used as a graphical interface. Three.js provides primitives that simplifies creating objects, such as a flat plane, that can be used in the creation of user menus or graphical displays. Three.js can also be used to display text in virtual reality, which will be helpful for buttons and help text.

\subsection{User Interface Functionality}
Many of the graphical user interfaces will need some sort of functionality, otherwise they will just be objects that do nothing. The main user interface functionality is menu functionality. The user must be able to interact with a menu to select an experiment station, pause and play objects or the environment, spawn or delete objects, and others. Nearly everything the user can interact with requires some sort of interface. We will use WebXR's controller API, its Magic Window feature, and Three.js to create our user interface.

\subsection{Hardware Interfacing}
\subsubsection{Head Mounted Display}
HMDs must be supported, therefore some sort of interfacing is needed to set up the peripherals such as the controllers and head tracking. This will be done using WebXR controller API. The WebXR controller API is a low level API that is built into WebXR. WebXR controller API will be used for finer details in creating specific controls and gamepad setup.

\subsubsection{Mobile Device}
Mobile devices should be supported, in particular those with Google Daydream support. This will also be done via the WebXR controller API. Touchscreen should also be supported for devices without Daydream support. This will be achieved by using various components with touchscreen support, such as a movement-controls component that allows the camera position to be moved by using a touchscreen with touch-drag support.

\section{Design Elements}

\subsection{Physics}
\subsubsection{Cannon.js}
\textbf{Type:} Physics Engine API\\
\textbf{Purpose:} Physics engine written in JavaScript. Animates objects in the scene with realistic physical movements.

\subsection{Graphics}
\subsubsection{Three.js}
\textbf{Type:} WebGL Framework\\
\textbf{Purpose:} Renders 3D content onto the site and connected devices for viewing.

\subsection{Web Hosting}
\subsubsection{Zeit Now}
\textbf{Type:} Global Serverless Deployments\\
\textbf{Purpose:} Zeit Now will be used for hosting static websites and able to edit, push, and make live changes directly from a GitHub repository. 

\subsection{Cloud Platform System}
\subsubsection{Now-Zeit}
\textbf{Type:} Cloud Platform\\
\textbf{Purpose:} Global Serverless Deployments. Now makes serverless application deployment easy.

\subsection{Version Control Systems}
\subsubsection{GitHub}
\textbf{Type:} Hosting Platform (Version Control)\\
\textbf{Purpose:} GitHub will manage code and track version control changes of the code. 

\subsection{Language}
\subsubsection{JavaScript}
\textbf{Type:} Language\\
\textbf{Purpose:} JavaScript will be used to write all client side code for the app.

\subsection{Asset Bundler}
\subsubsection{Parcel}
\textbf{Type:} Build tool (Asset Bundler)\\
\textbf{Purpose:} Parcel will be used to bundle all of our assets, including code, images, 3D object files, and textures.

\subsection{Continuous Integration}
\subsubsection{Circle CI}
\textbf{Type:} Continuous Integration\\
\textbf{Purpose:} Circle CI will be used to build and test our application after every push to GitHub and before each pull request is merged to maintain code quality.

\subsection{User Interface Visualization}
\subsubsection{Three.js}
\textbf{Type:} API\\
\textbf{Purpose:} Three.js will be used to create graphical models that will be used for user interfaces, such as menus or graphical displays.

\subsection{User Interface Functionality}
\subsubsection{JavaScript}
\textbf{Type:} Language\\
\textbf{Purpose:} JavaScript will be used to write functional code that performs requested actions from a user interacting with a menu.

\subsection{Hardware Interfacing}
\subsubsection{Gamepad API}
\textbf{Type:} API\\
\textbf{Purpose:} Gamepad API provides tools to connect and map controllers to perform specific actions.

\subsubsection{WebXR}
\textbf{Type:} API\\
\textbf{Purpose:}  WebXR provides capabilities for detecting connected hardware which is useful in determining compatibility.

\section{Design Rationale}

\subsection{Physics}
Cannon.js was chosen as the physics engine because of its enormous feature set and VR support. Cannon.js has an active community and extensive documentation. There are lots of examples with source code that could be helpful during implementation of this project.

\subsection{Graphics}
The decision to use Three.js is mostly due to its popularity as a 3D graphics API in the web development world. It has a proven record of quality and efficiency along with a large community that supports it.

\subsection{Web Hosting}
The web hosting services we chose to use is Zeit Now because it will offer us static web pages. A static web page is a web page that can deliver the users exactly what is stored, without any server side rendering required. Zeit Now will allow us to publish a new deployment of the website for each commit that is pushed to GitHub, so that we can share that specific version of the website with other teammates and our client. Now also has easy integration with GitHub repositories.

\subsection{Version Control Systems}
We will be using GitHub for our project repository because it will allow us to have Git repository hosting service and publishes web sites. GitHub host code repositories for all types of language, not just HTML and CSS. It also has a feature to make it easy to create a multi-file website hosted at GitHub Pages. GitHub also provides access control and collaboration features like fork, pull, and merge.  

\subsection{Cloud Platform System}
We will be using Now for global deployment network. It makes our team productive by removing servers and configuration, giving us a seamless developer experience to build modern scalable web application.

\subsection{Language}
We will be using JavaScript for the client side application because it is what is supported by WebXR and ThreeJS. Unfortunately, ThreeJS does not have updated TypeScript types and WebXR has no TypeScript type definitions at all, so its not practical to use for this project.

\subsection{Asset Bundler}
We chose to use Parcel to bundle our client-side assets because of how simple it is to set up. Parcel also is able to harmonize the different module formats including the old Common.js format which is the module format used for the Three.js examples.  We need may need the Three.js orbit controls or perhaps it's model loaders.  There is essentially no configuration - we will only need to include the root TypeScript file in our HTML page, and Parcel will figure out the dependencies based on our import statements. If we determine that we need more granularity of control in our configuration, we will switch to using Webpack, since it provides the same features, but requires (and allows) more configuration. 

\subsection{Circle CI}
We chose to use Circle CI because it is simple to configure and free for open source projects. It also has built in webhooks with GitHub, so it will be simple to configure it to build and test our application after pushing changes to GitHub and before merging pull requests. Travis CI is very similar in that it is free for open source projects and easily configurable, but build tasks tend to run somewhat slower on Travis CI than Circle CI.

\subsection{User Interface Visualization}
Three.js is used since it is a well regarded API for 3D modeling. The user interface will be using 3D modeling rather than some 2D user interface toolkit, such as Open Source Qt, because it will be in a VR environment. Menus and other graphical user interfaces cannot exist on the forefront of the screen, like it would on a normal program, in a VR simulation without causing strain on the user. This is because the user's eyes will be very close to the screens due to the nature of HMDs. Instead, spatial UI is needed, which is essentially UI that exists within the environment of the simulation. Therefore, some sort of 3D graphics modeling will be needed for creating the graphical user interfaces, in the case of this project being Three.js.

\subsection{User Interface Functionality}
User interface functionality is vital to the project working correctly. Without some sort of user interface functionality, it would be impossible for the user to interact with a lot of the entities in the simulation. Many things, such as changing the properties of objects, pausing an object or the environment, changing laws of physics, and spawning or deleting items depend on some sort of user interface. JavaScript in combination with the WebXR API and Three.js is how these functionalities will be created.

\subsection{Hardware Interfacing}
Hardware interfacing is an absolute necessity; without it, the user would be unable to do anything. The choice of supporting low end devices, such as mobile devices using Google Cardboard, stems from the goal of the project: providing an affordable method of physics education. Those who aren't able to afford the expensive materials required to perform some physics experiments likely won't be able to afford a 500 dollar VR headset, not to mention a computer that can effectively support it.  Despite the WebXR specification being fairly stable, it's implementations may not be so using any library between us and the API could block our development when changes require that library to issue a patch.  Additionally, we want to demonstrate the API's functionality so we choose to write our own layer ontop of the WebXR API.

\end{document}
